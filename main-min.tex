%% More detailed documentation and instructions can be found in README.md

%%%%% NOTICES %%%%%
%% This template or its associated class file don't
%% come with a warranty. The content is provided as is,
%% without even the implied promise of fitness to the
%% mentioned purpose. You, as the author of the thesis,
%% are responsible for the entire work, including the
%% provided material. No one else is liable to you for
%% any damage inflicted on you or your thesis, were it
%% caused by using this template or not.

%%%%% PREAMBLE %%%%%

%%%%% Document class declaration.
% The possible optional arguments are
%   finnish - thesis in Finnish (default)
%   english - thesis in English
%   numeric - citations in numeric style (default)
%   authoryear - citations in author-year style
% Example: \documentclass[english, authoryear]{tauthesis}
%          thesis in English with author-year citations
\documentclass{tauthesis}

% The glossaries package throws a warning:
% No language module detected for 'finnish'.
% You can safely ignore this. All other
% warnings should be taken care of!

%%%%% Your packages.
% Before adding packages, see if they can be found
% in tauthesis.cls already. If you're not sure that
% you need a certain package, don't include it in
% the document! This can dramatically reduce
% compilation time.

%%%%% Your commands.

% Print verbatim LaTeX commands
\newcommand{\verbcommand}[1]{\texttt{\textbackslash #1}}

%%%%% Glossary information.

% \loadglsentries[main]{tex/sanasto.tex} % Uncomment and add your glossary source
\makeglossaries

%%%%% Citation information.

\addbibresource{tex/references.bib}

\begin{document}

% Path for logo graphics
\graphicspath{{logo/}}

%%%%% FRONT MATTER %%%%%

\frontmatter

%%%%% Thesis information and title page.

% The titles of the work. If there is no subtitle,
% leave the arguments empty. Pass the title in
% the primary language as the first argument
% and its translation to the secondary language
% as the second.
% To leave a required field intentionally empty,
% replace the text with ~ (tilde)
\title{Kuvaava otsikko}{A Descriptive Title}
\subtitle{Tarkentava alaotsikko}{A Specifying Subtitle}

% The author name.
\author{Etunimi Sukunimi}

% The finishing date of the thesis (YYYY-MM-DD).
\finishdate{2019}{01}{17}

% The type of the thesis (e.g. Kandidaatintyö
% or Master of Science Thesis) in the primary
% and the secondary languages of the thesis.
\thesistype{Opinnäytetyön taso}{Thesis type}

% The faculty and degree programme names in
% the primary and the secondary languages of
% the thesis.
\facultyname{Tiedekunnan nimi}{Faculty Name}
\programmename{Tutkinto-ohjelma}{Degree Programme}

% The keywords to the thesis in the primary and
% the secondary languages of the thesis
\keywords%
    {avainsana, avainsana, avainsana, avainsana, avainsana}
    {keyword, keyword, keyword, keyword, keyword}

\maketitle

%%%%% Abstracts and preface.

% Write the abstract(s) and the preface
% into a separate file for the sake of clarity.
% Pass the appropriate file name as the first
% argument to these commands. Put the \abstract
% in the primary language first and the
% \otherabstract in the secondary language second.
% Those who do not speak Finnish only need the
% first abstract. The second argument of
% the \preface command takes the place where
% the thesis was signed in.
\abstract{tex/tiivistelma.tex}
\otherabstract{tex/abstract.tex}
\preface{tex/alkusanat.tex}{Tampereella}

%%%%% Table of contents.

\tableofcontents

%%%%% Lists of figures, tables, listings and terms.

% Print the lists of figures and/or tables.
% (Un)comment either of these commands as required.
% Both are optional, but if there are many important
% figures/tables, listing them may be a good idea.

\listoffigures
\listoftables

% Print the glossary of terms.

\glossary

%%%%% MAIN MATTER %%%%%

\mainmatter

% Write each of the chapters of the thesis
% into a separate file for the sake of clarity.
% They can be \input as shown below. Give both
% the chapters and their files as descriptive
% names as possible.
\chapter{Johdanto}
\label{ch:johdanto}

Tämä mallipohja liittyy Tampereen yliopiston tekniikan alan opinnäytteiden kirjoitusohjeisiin \parencite{kirjoitusohje2018}.
Tästä versiosta on poistettu kaikki mahdollinen, jotta jäljelle jäisi vain oleellinen.

This document template conforms to the Guide to Writing a Thesis in Tampere University \parencite{thesisguide2018}.
This version has been stripped as far as possible to leave only the essentials.


\chapter{Yhteenveto}
\label{ch:yhteenveto}
\input{tex/yhteenveto.tex}

%%%%% Bibliography/references.

% Print the bibliography according to the
% information in ./tex/references.bib and
% the in-line citations used in the body of
% the thesis.
\printbibliography[heading=bibintoc]

%%%%% Appendices.

% Use only if it clarifies the structure of
% the document. Remember to introduce each
% appendix and its content.

\begin{appendices}

\chapter{Esimerkkiliite}
\label{ch:liite}
\input{./tex/liite.tex}

\end{appendices}

\end{document}